\section{LibFinder}\label{sec:lib_finder}

If you want to use some libraries in your application, you have to configure your project to use them. Such configuration process may be hard and annoying because each library can use custom options scheme. Another problem is that configuration differs on platforms which result in incompatibility between unix and windows projects.

LibFinder provides two major functionalities:

\begin{itemize}
\item Searching for libraries installed on your system
\item Including library in your project with just few mouse clicks making project platform-independent
\end{itemize}

\subsection{Searching for libraries}

Searching for libraries is available under \menu{Plugins,Library finder} menu. It's purpose is to detect libraries installed on your system and store the results inside LibFinder's database (note that these results are not written into \codeblocks files). Searching starts with dialogue where you can provide set of directories with installed libraries. LibFinder will scan them recursively so if you're not sure you may select some generic directories. You may even enter whole disks -- in such case searching process will take more time but it may detect more libraries (see \pxref{fig:list_of_directories}).

\screenshot{list_of_directories}{List of directories}

When LibFinder scans for libraries, it uses special rules to detect presence of library. Each set of rules is located in xml file. Currently LibFinder can search for wxWidgets 2.6/2.8, \codeblocks SDK and GLFW -- the list will be extended in future.

\hint{To get more details on how to add library support into LibFinder, read \file{src/plugins/contrib/lib\_finder/lib\_finder/readme.txt} in \codeblocks sources.}

After completing the scan, LibFinder shows the results (see \pxref{fig:search_results}).

\screenshot{search_results}{Search results}

In the list you check libraries which should be stored into LibFinder's database. Note that each library may have more than one valid configuration and settings added ealier are more likely to be used while building projects.

Below the list you can select what to do with results of previous scans:

\begin{description}
\item[Do not clear previous results] This option works like an update to existing results -- it adds new ones and updates those which already exist. This option is not recommended.
\item[Second option (Clear previous results for selected libraries)] will clear all results for libraries which are selected before adding new results. This is the recommended option.
\item[Clear all previous library settings] when you select this option, LibFinder's database will be cleared before adding new results. It's useful when you want to clean some invalid LibFinder's database.
\end{description}

Another option in this dialogue is \menu{Set up Global Variables}. When you check this option, LibFinder will try automatically configure Global Variables which are also used to help dealing with libraries.

If you have pkg-config installed on your system (it's installed automatically on most linux versions) LibFinder will also provide libraries from this tool. There is no need to perform scanning for them -- they are automatically loaded when \codeblocks starts.

\subsection{Including libraries in projects}

LibFinder adds extra tab in Project Properties \menu{Libraries} -- this tab shows libs used in project and libs known in LibFinder. To add library into your project, select it in right pane and click $<$ button. To remove library from project, select it on the left pane and click $>$ button (see \pxref{fig:project_configuration}).

\screenshot{project_configuration}{Project configuration}

You can filter libraries known to LibFinder by providing search filter. The \menu{Show as Tree} checkbox allows to switch between categorized and uncategorized view.

If you want to add library which is not available in LibFinder you may use \menu{Unknown Library} field. Note that you should enter library's shortcode (which usually matches global variable name) or name of library in pkg-config. E.g.

\begin{code}
$(#GLOBAL_VAR_NAME.include)
\end{code}

Checking the \menu{Don't setup automatically} option will notify LibFinder that it should not add libraries automatically while compiling this project. In such case, LibFinder can be invoked from build script. Example of such script is generated and added to project by pressing \menu{Add manual build script}.

\subsection{Using LibFinder and projects generated from wizards}

Wizards will create projects that don't use LibFinder. To integrate them with this plugin, you will have to manually update project build options. This can be easily achieved by removing all library-specific settings and adding library through \menu{Libraries} tab in project properties.

Such project becomes cross-platform. As long as used libs are defined in LibFinder's database, project's build options will be automatically updated to match platform-specific library settings.


