\section{Source Code Exporter}

Oft ergibt sich die Notwendigkeit, den Quelltext in andere Anwendungen oder in Emails zu übernehmen. Beim schlichten Kopieren des Textes geht jedoch die Formatierung verloren, was den Text sehr unübersichtlich macht.
Die Export Funktion in \codeblocks schafft hier Abhilfe. Über \menu{File,Export} kann ein gewünschtes Dateiformat für die Exportdatei ausgewählt werden. Danach übernimmt das Programm den Dateinamen und das Zielverzeichnis der geöffneten Quelldatei und schlägt diesen als Name zum speichern vor. Die jeweilige Dateiendung wird durch das Exportformat bestimmt. Es stehen folgende Formate zur Verfügung.

\begin{description}
\item[html] Ein textbasiertes Format, das in einem Web-Browser oder Anwendungen zur Textverarbeitung angezeigt werden kann.
\item[rtf] Das Rich Text Format ist ein textbasiertes Format, das sich in Programmen zur Textverarbeitung wie Word oder OpenOffice öffnen lässt.
\item[odt] Open Document Text Format ist ein standardisiertes Format, dass von Sun und O'Reilly festgelegt wurde. Dieses Format kann von Word, OpenOffice und anderen Textverarbeitungsprogrammen eingelesen werden.
\item[pdf] Das Portable Document Format kann mit Anwendungen wie Acrobat Reader geöffnet werden.
\end{description}
