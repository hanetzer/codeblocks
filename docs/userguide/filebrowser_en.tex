\section{File Explorer and Shell Extension Plugin}\label{sec:file_explorer}

The File Explorer \pxref{fig:file_explorer} is included in the Shell Extensions plugin, and can be found in the \samp{Files} tab. The composition of the File Explorer is shown in \pxref{fig:file_explorer}.

On top you will find a field for entering the path. By clicking the button at the end of this field, the drop-down field will list a history of the past entries which can be navigated via a scroll bar. The up arrow key on the right-hand side of the field moves up the directory structure one directory.

In the \samp{Wildcard} field you can enter a filter term for the file display. Leaving the field empty or entering \codeline{*} results in all files being displayed. Entering \codeline{*.c;*.h}, for example will result in solely C sources and header files being displayed. Opneing the pull-down field will, again, list a history of the last entries.

\figures[hbt!]{file_explorer}{The file manager}

Pressing the Shift key and clicking selects a group of files or directories, pressing the Ctrl key and clicking selects multiple separate files or directories.

The following operations can be started via the context menu if one or multiple directories are selected in the File Explorer:

\begin{description}
\item[Make Root] ...defines the current directory as the root directory.
\item[Add to Favorites] ...sets a marker for the directory and stores it as a favourite. This function allows you to navigate quickly between frequently used directories, also on different network drives.
\item[New File] ...creates a new file in the selected directory.
\item[Make Directory] ...creates a new subdirectory in the selected directory.
\end{description}

The following operations can be started via the context menu if one or multiple files or directories are selected in the File Explorer:

\begin{description}
\item[Duplicate] ...copies a file/directory and renames it.
\item[Copy To] ...opens a dialogue for entering the target directory in which the copied file/directory is to be stored.
\item[Move To] ...moves the selection to the target location.
\item[Delete] ...deletes the selected files/directories.
\item[Show Hidden Files] ...activates/deactivates the display of hidden system files. When activated, this menu entry is checkmarked.
\item[Refresh] ...actualises the display of the directory tree.
\end{description}

The following operations can be started via the context menu if one or multiple files are selected in the File Explorer:

\begin{description}
\item[Open in CB Editor] ...opens the selected file in the \codeblocks editor.
\item[Rename] ...renames the selected file.
\item[Add to active project] ...adds the file(s) to the active project.
\end{description}

\hint{The files/directories selected in the File Explorer can be accessed in the Shell Extensions plugin via the \codeline{mpaths} variable.}

User-defined functions can be specified via the menu command \menu{Settings,Environment,Shell Extensions}. In the Shell Extensions mask, a new function which can be named at random, is created via the \samp{New} button. In the \samp{ShellCommand Executable} field, the executable program is stated, and in the field at the bottom of the window, additional parameters can be passed to the program.
By clicking the function in the context menu or the Shell Extensions menu, the function is started and will then process the selected files/directories. The output is redirected to a separate shell window.

For example a menu entry in \menu{Shell Extensions,SVN} and in the context menu is created for \samp{SVN}. \codeline{$file} in this context means the file selected in the File Explorer, \codeline{$mpath} the selected files or directories (see \pxref{sec:builtin_variables}).

\begin{code}
 Add;$interpreter add $mpaths;;;
\end{code}

This and every subsequent command will create a submenu, in this case called \menu{Extensions,SVN,Add}. The context menu is extended accordingly. Clicking the command in the context menu will make the SVN command \codeline{add} process the selected files/directories.

TortoiseSVN is a widespread SVN program with integration in the explorer. The program \file{TortoiseProc.exe} of TortoiseSVN can be started in the command line and dispalys a dialog to collect user input. So you can perform the commands, that are available as context menu in the explorer also in the command line. Therefore you can integrate it also a shell extension in \codeblocks. For example the command

\begin{cmd}
TortoiseProc.exe /command:diff /path:$file
\end{cmd}

will diff a selected file in the \codeblocks file explorer with the SVN base. See \pxref{fig:interpreter} how to integrate this command.
\hint{For files that are under SVN control the file explorer shows overlay icons if they are actived via menu \menu{View,SVN Decorators}.}

\screenshot{interpreter}{Add a shell extension to the context menu}

\genterm{Example}

You can use the file explorer to diff files or directories. Follow these steps:

\begin{enumerate}
\item Add the name via menu \menu{Settings,Environment,Shell Extensions}. This is shown as entry in the interpreter menu and the context menu.
\item Select the absolute path of Diff executable (e.g. kdiff3). The program is accessed with the variable \codeline{$interpreter}.
\item Add parameters of the interpreter
\begin{cmd}
Diff;$interpreter $mpaths;;;
\end{cmd}
\end{enumerate}

This command will be executed using the selected files or directories as parameter. The selection is accessed via the variable \codeline{$mpaths}. This is an easy way to diff files or directories.

\hint{The plugin support the use of \codeblocks variables within the shell extension.}

% Actions string format: Name;Command;[W|C];WorkDir;EnvVarSet
% (the last two ; delimit settings for the working directory and (not implemented) environment variable set)
%
\begin{codeentry}
\item[\$interpreter] Call this executable.% Aufzurufendes Programm
\item[\$fname] Name of the file without extension. %Name der Datei ohne Endung.
\item[\$fext] Extension of the selected file. %Dateiendung der ausgewählten Datei.
\item[\$file] Name of the file. %Dateiname mit Endung.
\item[\$relfile] Name of the file without path info. %Dateiname ohne Pfadangabe.
\item[\$dir] Name of the selected directory. %Ordnername mit Pfadangabe.
\item[\$reldir] Name of directory without path info. %Ordnername ohne Pfadanabe.
\item[\$path] Absolute path. %Absolute Pfad.
\item[\$relpath] Relative path of file or directory. %Relativer Pfad einer Datei oder Verzeichnis.
\item[\$mpaths] List of current selected files or directories. %Liste der ausgewählten Dateien oder Ordner.
\item[\$inputstr\{<msg>\}] String that is entered in a message window. %Zeichenkette die durch eine Eingabeaufforderung eingegeben wird.
\item[\$parentdir] Parent directory (../). %Übergeordnetes Verzeichnis (../)
\end{codeentry}

\hint{The entries of shell extension are also available as context menu in the \codeblocks editor.}
%Support for personalities.
%Bsp:
%%View;latex $fname.$fext;W;$parentdir
%
%\subsection{Support for Version Control Systems}
%
%Context menu \menu{View, SVN Decorators}
% Run the processes using option 'W' in the action string (to run an interpreter in the cbconsole runnner use 'C' in the action string). for example a python run action string to run a script in a dockable winodw tab might look like this:
%
% \begin{code}
% 'Run;$interpeter -u $file;W;;'
% \end{code}
%
% Command line variables:
% \begin{code}
% $interpreter, $file, $dir, $path, $mpaths
% Working directory variables: $dir, $parentdir
% \end{code}
%
