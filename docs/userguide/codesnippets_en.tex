\section{CodeSnippets}\label{sec:codesnippets}

The CodeSnippets plug-in makes it possible to structure text modules and links to files according to categories in a tree view. The modules are used for storing often used files and constructs in text modules and managing them in a central place. Imagine the following situation: A number of frequently used source files are stored in different directories of the file system. The CodeSnippets window provides the opportunity to create categories, and below the categories, links to the required files. With these features, you can control the access to the files independently from where they are stored within the file system, and you can navigate quickly between the files without the need to search the whole system.

\hint{You can use \codeblocks variables or environment variables in file links e.g. \codeline{$(VARNAME)/name.pdf} to parametrise a link in the CodeSnippets browser.}

The list of text modules and links can be stored in the CodeSnippets window by right-clicking and selecting \samp{Save Index} from the context menu. The file \file{codesnippets.xml} which will be created by this procedure, can then be found in the \file{codeblocks} subdirectory of your \file{Documents and Settings\osp Application data} directory. Under Linux, this information is stored in the \file{.codeblocks} subdirectory of your HOME directory. The \codeblocks configuration files will be loaded during the next start-up. If you wish to save the content of CodeSnippets at a different location, select the \samp{Save Index As} entry. To load this file, select \samp{Load Index File} during the next start-up of \codeblocks or include the directory in the \samp{Settings} context menu under \samp{Snippet Folder}. The settings are saved in the corresponding file \file{codesnippets.ini} in your application data.

For including a category, use the \samp{Add SubCategory} menu. A category can contain Snippets (text modules) or File Links. A text module is created via the \samp{Add Snippet} command in the context menu. The content is integrated into the text module as \samp{New snippet} by selecting the text passage in the \codeblocks editor and dragging and dropping it onto the module and the properties dialog pops up. Double-clicking the newly included entry or selecting \samp{Edit Text} will open an editor for the content.

\screenshot{edit_snippet}{Editing a text module}

Output of a text module is handled in \codeblocks via the context menu command \samp{Apply} or by dragging and dropping into the editor. Under Windows, the contents of a Snippet can also be dragged and dropped into other applications. In the CodeSnippets Browser you can copy a selected item with drag and drop to a different category.

Beyond this, text modules can be parametrised by \var{name} variables which can be accessed via \codeline{$(name)} (see \pxref{fig:edit_snippet}). The values of the variables can be retrieved in an entry field if the text module is called via the context menu command \samp{Apply}.

Besides the text modules, links to files can also be created. If, after having created a text module, you click the context menu command \samp{Properties}, then you can select the link target by clicking the \samp{Link target} button. This procedure will automatically convert the text module into a link to a file. In CodeSnippets, all text modules will be marked by a T symbol, links to a file by an F symbol and urls by an U symbol. If you want to open a selected file (link) in the codesnippets view just select the context menu \menu{Open File} or hold the \samp{Alt} key and make a double click on the file.

\hint{You can add even url (e.g. http://www.codeblocks.org) in text modules. The url will be opened in your favorite browser with the context menu \menu{Open Url}.}
%\hint{If you have chosen the setting \samp{open it with the associated application} under \menu{Settings,Environment} for file extension handling, the application assigned by Windows for the file extension will be used (see \pxref{sec:file_extension}).}

With this setting, if open a link to a pdf file from the codesnippets view a pdf viewer will be started automatically. This method makes it possible for a user to access files which are spread over the whole network, such as cad data, layouts, documentations etc., with the common applications, simply via the link. The content of the codesnippets is stored in the file \file{codesnippets.xml}, the configuration is stored in the file \file{codesnippets.ini} in your \file{application data} directory. This ini file will, for example, contain the path of the file \file{codesnippets.xml}.

\codeblocks supports the usage of different profiles. These profiles are called personalities. Starting \codeblocks with the command line option \opt{--personality=\var{profile}} will create a new or use an existing profile. Then the settings will not be stored in the file \file{default.conf}, but in \file{\var{personality}.conf} in your \file{application data} directory instead. The Codesnippets plugin will then store its settings in the file \file{\var{personality}.codesnippets.ini}. Now, if you load a new content \file{\var{name.xml}} in the Codesnippets settings via \samp{Load Index File}, this content will be stored in the corresponding ini file. The advantage of this method lies in the fact that in case of different profiles, different configurations for text modules and links can be managed.

The plug-in offers an additional search function for navigating between the categories and Snippets. The scope for searching Snippets, categories or Snippets and categories can be adjusted. By entering the required search expression, the corresponding entry is automatically selected in the view. \pxref{fig:codesnippets} shows a typical display in the CodeSnippets window.

\figures[hbt!][width=.4\columnwidth]{codesnippets}{CodeSnippets View}

\hint{When using voluminous text modules, the content of these modules should be saved in files via \samp{Convert to File Link} in order to reduce memory usage within the system. If you delete a codesnippet or file link it will be moved to the category \file{.trash}; if you hold the Shift key the item will be deleted.}
